%%% Local Variables: 
%%% mode: latex
%%% TeX-master: t
%%% End: 

%\documentclass[10pt,draft]{article}
\documentclass[10pt]{article}

\usepackage[T1]{fontenc}
\usepackage{textcomp}
\usepackage{a4wide}
\usepackage[monochrome]{color}
%\usepackage[colorlinks=true,pdftex,bookmarksnumbered,bookmarksopen]{hyperref}
\usepackage[colorlinks=false,pdftex,bookmarksnumbered,bookmarksopen]{hyperref}
\usepackage{float} 
%\usepackage{wrapfig}
%\usepackage{german}
%\usepackage[ngerman]{babel}
\usepackage[latin1]{inputenc}
%\usepackage{luximono}
%\usepackage{listings}
%\usepackage{ams,amsmath}
%\usepackage{pandora}
%\usepackage{pgf,tikz}
%\usepackage{fixme}

\begingroup
  \makeatletter
  \def\x#1\refstepcounter#2\@nil{%
    \endgroup
    \def\caption{#1\H@refstepcounter#2}%  
  }%
\expandafter\x\caption\@nil

%\lstset{
%  language=c++,
%  basicstyle=\footnotesize\color{black}\ttfamily,
%  keywordstyle=\color{blue},
%  showspaces=false,
%  showtabs=false
%}

% Use a small font for the verbatim environment
\makeatletter  % makes '@' an ordinary character
\renewcommand{\verbatim@font}{%
  \ttfamily\footnotesize\catcode`\<=\active\catcode`\>=\active%
}
\makeatother   % makes '@' a special symbol again

\author{Johannes Rudolph}
\title{Safer Java Bytecode Generation: A Scala Bytecode Library\\
Diploma thesis proposal}
\makeindex

\pagestyle{headings}

\begin{document}

\maketitle

\begin{center}
\small 
Albert-Ludwigs-Universit�t Freiburg\\
Arbeitsbereich Programmiersprachen\\
Prof. Thiemann\\
\end{center}

\section{Description}
To compile programs targetting the Java Virtual Machine(JVM), compiler writers
have to emit Java bytecodes, a serialized, binary form of the program. For this
task several libraries exist today. To enforce the well behaviour of the code,
the JVM specification defines several constraints which it imposes upon
bytecode.  The well-formedness of the generated code is then checked at
the time the JVM loads the code. \\\\
Classes being rejected by the classloader are undesirable, therefore
many bytecode-generating libraries aid the compiler writer by giving
him tools to check if the generated code in spe is
valid\cite{asmref}\footnote{\url{http://asm.objectweb.org/asm31/javadoc/user/org/objectweb/asm/tree/analysis/Analyzer.html}}. Using
such a tool errors are reported at the runtime of the compiler at
least. \\\\
I propose a Scala library, which - in the best case - doesn't even allow
illegal bytecode sequences to be represented in terms of the library. The
library makes use of Scala's type system to let the Scala compiler reject
illegal combinations of bytecode operations. This is accomplished by encoding
the state of the stack and the local variables as type parameters of an object
representing the current state of the frame. By defining bytecode operations as
transitions between one type of frame and another one can enforce that only
matching operations can be applied to a given frame. \\\\
As an application of the library a tool will be developed which compiles an
XPath expression into corresponding Java bytecodes which internally traverse the
DOM of an XML document.

\section{Procedure}
Bird's eye view:
\begin{enumerate}
\item Define the external interface of the library
\item Implement the interface: use ASM\cite{asmref} as backend to emit bytecodes
\item Use the library to implement the sample application
\end{enumerate}
The main work will be to define the interface since it has to contain
all the type information which provides the compile time safety later
on. As outlined in the structure below there are several steps
necessary in defining the interface itself. \\\\
The first step is to find an appropriate representation of the frame
at a given point in the execution. The state of a frame consists of
the types of the values on the stack as well as the types of local
variables. One can model such a state by using type parameters
recursively to build up a stack of types representing the types of
stack and local variable values. \\\\
Expressed with Scala traits it looks basically like this (only stack
state):
\begin{verbatim}
trait Stack{}
trait Nil extends Stack{}
trait S[T,R <: Stack]{}
\end{verbatim}
A stack consisting of 3 values of type int,String,float (from top to
bottom) would be
represented as
\begin{verbatim}
S[int,S[String,S[float,Nil]]]
\end{verbatim}
In a second step one has to find proper representation for the
bytecode operations. One can think of these operations as
transformations of the state of the frame and so they will be defined
as functions over specific input states mapping to output states. As
an example think of the operation \verb!iadd! which pops two integers
from the top of the stack and pushes their sum on top of the stack.
This operation could be defined as:
\begin{verbatim}
def iadd[R]: S[int,S[int,R]] => S[int,R]
\end{verbatim}
Such a definition of the operations has value in itself by being a
concise documentation of how an operations change the types of the
stack and local variables values.\\\\
One important aspect is how method calls can be represented in a type
safe manner. There are two cases: Either the method is already
accessible at compile time of the compiler or it will be defined in
terms of generated bytecodes.  scala.reflect.Code of the Scala library
provides a possibility to access the Scala AST of a given expression
and as such can be used to refer to existing methods. Only later
generated methods have to be forward declared before they can be
referenced.\\\\


\section{Proposed thesis structure}
\begin{enumerate}
\item Bytecode generation example (XPath or simple expression language)
\item Summary of "Constraints on Java Virtual Machine Code"\cite[�4.8]{jvm}
\item Bytecode DSL
  \begin{enumerate}
  \item The Frame type: Building up stacks of types to represent stack and local variables of a method
  \item Encoding of simple bytecode operations
  \item Method calls
    \begin{enumerate}
    \item Using scala.reflect.Code to access information about existing methods
    \item Forward references for code being defined (not yet sure how to do it)
    \end{enumerate}  
  \item Jumps: encoding of code references
  \item (Translating Scala compiler errors into human readable form)
  \end{enumerate}
\item Application: The XPath compiler
  \begin{enumerate}
  \item How it is done
  \item Some performance measurements
  \end{enumerate}
\item Discussion
\end{enumerate}

\section{Roadmap}
\begin{tabular}{l l}
1-15 July & Defining the core classes (subset of opcodes)\\
31 July & Working byte code generation for that subset\\
Aug & XPath Compilation\\
Sep & XPath performance tests, adding missing opcodes, cleanup of the library\\
Oct & Writing Chap 2,3\\
Nov & Writing Chap 1,4,5\\
Dec & Finishing: proof-reading etc, buffer for unexpected events...\\
\end{tabular}

\section{Some links}

\paragraph{Bytecode Libraries}
\begin{itemize}
\item A list of libraries: \url{http://java-source.net/open-source/bytecode-libraries}
\item \url{http://elliotth.blogspot.com/2008/03/generating-jvm-bytecode.html}
\end{itemize}

\paragraph{XPath}
\begin{itemize}
\item \url{http://www.asciiarmor.com/2004/11/13/java-xpath-10-engine-comparison-performance/}
\item \url{http://www.ibm.com/developerworks/library/x-javaxpathapi.html#listing3}
\item \url{http://dom4j.org/benchmarks/xpath/index.html}
\item \url{http://www.nearinfinity.com/blogs/page/sfarley?entry=gpath_versus_xpath}
\end{itemize}

\bibliographystyle{plaindin}
\bibliography{bytecode}

\end{document}

