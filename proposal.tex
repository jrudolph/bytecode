%%% Local Variables: 
%%% mode: latex
%%% TeX-master: t
%%% End: 

%\documentclass[10pt,draft]{article}
\documentclass[10pt]{article}

\usepackage[T1]{fontenc}
\usepackage{textcomp}
\usepackage{a4wide}
\usepackage[monochrome]{color}
%\usepackage[colorlinks=true,pdftex,bookmarksnumbered,bookmarksopen]{hyperref}
\usepackage[colorlinks=false,pdftex,bookmarksnumbered,bookmarksopen]{hyperref}
\usepackage{float} 
%\usepackage{wrapfig}
%\usepackage{german}
\usepackage[ngerman]{babel}
\usepackage[latin1]{inputenc}
%\usepackage{luximono}
\usepackage{listings}
%\usepackage{ams,amsmath}
%\usepackage{pandora}
\usepackage{pgf,tikz}
\usepackage{fixme}

\begingroup
  \makeatletter
  \def\x#1\refstepcounter#2\@nil{%
    \endgroup
    \def\caption{#1\H@refstepcounter#2}%  
  }%
\expandafter\x\caption\@nil

\lstset{
  language=c++,
  basicstyle=\footnotesize\color{black}\ttfamily,
  keywordstyle=\color{blue},
  showspaces=false,
  showtabs=false
}

% Use a small font for the verbatim environment
\makeatletter  % makes '@' an ordinary character
\renewcommand{\verbatim@font}{%
  \ttfamily\footnotesize\catcode`\<=\active\catcode`\>=\active%
}
\makeatother   % makes '@' a special symbol again

\author{Johannes Rudolph}
\title{Safer Java Bytecode Generation: A Scala Bytecode Library\\
Diploma thesis proposal}
\makeindex

\pagestyle{headings}

\begin{document}

\maketitle

\begin{center}
\small 
Albert-Ludwigs-Universit�t Freiburg\\
Arbeitsbereich Programmiersprachen\\
Prof. Thiemann\\
\end{center}

\section{Abstract}
To compile programs targetting the Java Virtual Machine(JVM), compiler writers
have to emit Java bytecodes, a serialized, binary form of the program. For this
task several libraries exist today. To enforce the well behaviour of the code,
the JVM specification defines several constraints which it imposes upon
bytecode.  The well-formedness of the generated code is then checked at
the time the JVM loads the code. 

Classes being rejected by the classloader are undesirable, therefore
many bytecode-generating libraries aid the compiler writer by giving
him tools to check if the generated code in spe is
valid\cite{asmref}\footnote{\url{http://asm.objectweb.org/asm31/javadoc/user/org/objectweb/asm/tree/analysis/Analyzer.html}}. Using
such a tool errors are reported at the runtime of the compiler at
least.

I propose a Scala library, which - in the best case - doesn't even allow
illegal bytecode sequences to be represented in terms of the library. The
library makes use of Scala's type system to let the Scala compiler reject
illegal combinations of bytecode operations. This is accomplished by encoding
the state of the stack and the local variables as type parameters of an object
representing the current state of the frame. By defining bytecode operations as
transitions between one type of frame and another one can enforce that only
matching operations can be applied to a given frame.

As an application of the library a tool will be developed which compiles an
XPath expression into corresponding Java bytecodes which internally traverse the
DOM of an XML document.

\section{Proposed thesis structure}
\begin{enumerate}
\item Bytecode generation example (XPath or simple expression language)
\item Summary of "Constraints on Java Virtual Machine Code"\cite[�4.8]{jvm}
\item Bytecode DSL
  \begin{enumerate}
  \item The Frame type: Building up stacks of types to represent stack and local variables of a method
  \item Encoding of simple bytecode operations
  \item Method calls
    \begin{enumerate}
    \item Using scala.reflect.Code to access information about existing methods
    \item Forward references for code being defined (not yet sure how to do it)
    \end{enumerate}  
  \item Jumps: encoding of code references
  \item (Translating Scala compiler errors into human readable form)
  \end{enumerate}
\item Application: The XPath compiler
  \begin{enumerate}
  \item How it is done
  \item Some performance measurements
  \end{enumerate}
\item Discussion
\end{enumerate}

\section{Roadmap}
\begin{tabular}{l l}
1-15 July & Defining the core classes (subset of opcodes)\\
31 July & Working byte code generation for that subset\\
Aug & XPath Compilation\\
Sep & XPath performance tests, adding missing opcodes, cleanup of the library\\
Oct & Writing Chap 2,3\\
Nov & Writing Chap 1,4,5\\
Dec & Finishing: proof-reading etc, buffer for unexpected events...\\
\end{tabular}

\section{Some links}

\paragraph{Bytecode Libraries}
\begin{itemize}
\item A list of libraries: \url{http://java-source.net/open-source/bytecode-libraries}
\item \url{http://elliotth.blogspot.com/2008/03/generating-jvm-bytecode.html}
\end{itemize}

\paragraph{XPath}
\begin{itemize}
\item \url{http://www.asciiarmor.com/2004/11/13/java-xpath-10-engine-comparison-performance/}
\item \url{http://www.ibm.com/developerworks/library/x-javaxpathapi.html#listing3}
\item \url{http://dom4j.org/benchmarks/xpath/index.html}
\item \url{http://www.nearinfinity.com/blogs/page/sfarley?entry=gpath_versus_xpath}
\end{itemize}

\bibliographystyle{plaindin}
\bibliography{bytecode}

\end{document}

